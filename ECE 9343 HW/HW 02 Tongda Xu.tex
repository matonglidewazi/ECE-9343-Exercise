\documentclass[]{article}
\usepackage{graphicx}
\usepackage{indentfirst}

\usepackage{clrscode3e}

%clrs code is good!

\title{HW02 for ECE 9343}
\author{Tongda XU, N18100977}

\begin{document}

\maketitle

\section{Question 1: 3-divide maximum subarray}

\begin{codebox}
	\Procname{$\proc{maxFromLeft($A, p, r$)}$}
	\li $max = -\infty$
	\li \For $i \gets p$ \To $r$
	\li		\Do $max = Sum(A,p,i)>max?Sum(A,p,i):max$
 		\End
	\li \Return $max$
\end{codebox}

\begin{codebox}
	\Procname{$\proc{maxFromRight($A, p, r$)}$}
	\li $max = -\infty$
	\li \For $i \gets r$ \Downto $p$
	\li		\Do $max = Sum(A,i,r)>max?Sum(A,i,r):max$
	\End
	\li \Return $max$
\end{codebox}

\begin{codebox}
	\Procname{$\proc{THREE-FOLD-MAXSUB($A, p, r$)}$}
	\li $s = \left \lfloor (p+r)/3 \right \rfloor$
	\li $t = \left \lfloor (p+r)2/3 \right \rfloor$
	\li \If $Sum(A,s,t-1) > 0$
	\li 	\Then \Return $max(maxFromLeft(A, p, s-1), maxFromRight(A, t, r)) + Sum(A,s,t-1)$
	\li	\Else \Return $max(maxFromLeft(A, p, s-1), maxFromRight(A, t, r))$
	\End
\end{codebox}

The time complexity is $\Theta(n)$

\section{Question 2: Intermediate Sequence}

\begin{codebox}
	\Procname{$\proc{bubble sort($A$)}$}
	\li $A = [11, 8, 7, 5, 3, 1] $
	\li $\rightarrow [8, 11, 7, 5, 3, 1] \rightarrow [8, 7, 11, 5, 3, 1] \rightarrow [8, 7, 5, 11, 3, 1] 
	\rightarrow [8, 7, 5, 3, 11, 1] \rightarrow [8, 7, 5, 3, 1, 11]$
	\li $\rightarrow [7, 8, 5, 3, 1, 11] \rightarrow [7, 5, 8, 3, 1, 11] \rightarrow [7, 5, 3, 8, 1, 11]\rightarrow [7, 5, 3, 1, 8, 11]$
	\li $\rightarrow [5, 7, 3, 1, 8, 11] \rightarrow [5, 3, 7, 1, 8, 11] \rightarrow [5, 3, 1, 7, 8, 11]$	
	\li $\rightarrow [3, 5, 1, 7, 8, 11]\rightarrow [3, 1, 5, 7, 8, 11] $
	\li $\rightarrow [1, 3, 5, 7, 8, 11]$
\end{codebox}

\begin{codebox}
	\Procname{$\proc{insertion sort($A$)}$}
	\li $A = [11, 8, 7, 5, 3, 1] $
	\li $\rightarrow [8, 11, 7, 5, 3, 1]$
	\li $\rightarrow [8, 7, 11, 5, 3, 1] \rightarrow [7, 8, 11, 5, 3, 1]$
	\li $\rightarrow [7, 8, 5, 11, 3, 1] \rightarrow [7, 5, 8, 11, 3, 1] \rightarrow [5, 7, 8, 11, 3, 1]$
	\li $\rightarrow [5, 7, 8, 3, 11, 1] \rightarrow [5, 7, 3, 8, 11, 1] \rightarrow [5, 3, 7, 8, 11, 1] \rightarrow [3, 5, 7, 8, 11, 1]$
	\li $\rightarrow [3, 5, 7, 8, 1, 11] \rightarrow [3, 5, 7, 1, 8, 11] \rightarrow [3, 5, 1, 7, 8, 11] 
	\rightarrow [3, 1, 5, 7, 8, 11] \rightarrow [1, 3, 5, 7, 8, 11]$
\end{codebox}

\section{Question 3: Illustrate Merge Sort}

\begin{codebox}
	\Procname{$\proc{Merge sort($A$)}$}
	\li $15, 16, 25, 29, 30, 40, 48$
	\li $15, 29, 48 || 16, 25, 30, 40$
	\li $29 || 15, 48 || 25, 40 || 16, 30$
	\li $- || 48 || 15 || 40 || 25 || 16 || 30$
\end{codebox}

\section{Question 4: CLRS Problem 2-1}
\subsection{a. show time complexity}
$\Theta(T) = \frac{n}{k}\Theta(n^{2}) = \Theta(nk)$
\subsection{b. show merge, c. show whole}

There should not be anything special about Merge function, just use the original interface and implement of Merge in CLRS pp 31.\\
$$ T(n)=\left\{
\begin{array}{lcl}
 n       &      & {n \le k}\\
2T(\frac{1}{2}n) + n     &      & {n > k}\\
\end{array} \right. $$

Regarding the iterative tree, it is easy to notice that:
For branch (Merge), the complexity is $\Theta(nlg\frac{n}{k})$, For leaf (Insertion sort), is $\Theta(nk)$
\begin{codebox}
	\Procname{$\proc{Merge-sort($A, p, r, k$)}$}
	\li \If $r-p + 1 \le k$
	\li \Then   Insertion-Sort$(A,p,r)$
	\li 		\Return
	\li \ElseIf $p < r$
	\li \Then	$q = \left \lfloor (p+r)/2 \right \rfloor$
	\li 		Merge-Sort($A, p, q$)
	\li			Merge-Sort($A, q+1, r$)
	\li			Merge ($A, p, q, r$)
	\li			\Return
	\li \Else  \Return
		\End
\end{codebox}

\section{Question 5: CLRS Problem 6.1-3}

\section{Question 6: CLRS Problem 6.2-6}

1. Note that the height of a Heap is no more than $lg(n+\frac{1}{2}n - 1)$ in worst condition\\
2. Note that each round of $MAX-HEAPIFY$ takes constant time\\
4. Each time $MAX-HEAPIFY$ happen, the height of pointer $\leftarrow$ pointer- 1\\
5. We have:\\ 
	$$ T(h)=\left\{
	\begin{array}{lcl}
	c       &      & {h = 0}\\
	T(h - 1) + c     &      & {n > 0}\\
	\end{array} \right. $$\\

Solves: $T(h) = \Theta(h) = \Omega((lg\frac{3}{2}n - 1) = \Omega (lgn)$


\end{document}
