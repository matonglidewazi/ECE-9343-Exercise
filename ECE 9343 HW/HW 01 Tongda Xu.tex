\documentclass[]{article}
\usepackage{graphicx}
%opening
\title{HW01 for ECE 9343}
\author{Tongda XU, N18100977}

\begin{document}

\maketitle

\section{Question 1: Prove the Symmetry property}

$f(n) = \Theta(g(n)) \rightarrow \exists c_{1}, c_{2}, n_{0}, \forall n > n_{0}, 0 \le c_{1}g(n) \le f(n) \le c_{2}g(n)\\
\leftrightarrow \forall n > n_{0}, 0 \le \frac{f(n)}{c_{2}} \le g(n) \le \frac{f(n)}{c_{1}}\\
\leftrightarrow g(n) = \Theta(f(n))$

\section{Question 2: Problem 3-2}

\begin{tabular}{c c c c c c c}

	A & B & O & o & $\Omega$ & $\omega$ & $\Theta$ \\ 
	\hline 
	$lg^kn$ & $n^\epsilon$ & yes & yes & no & no & no \\ 
	\hline 
	$n^k$ & $c^n$ & yes & yes & no & no & no \\ 
	\hline 
	$n^{\frac{1}{2}}$ & $n^{sinn}$ & no & no & no & no & no \\ 
	\hline 
	$2^{n}$ & $2^{\frac{1}{2}n}$ & no & no & yes & yes & no \\ 
	\hline 
	$n^{lgc}$ & $c^{lgn}$ & yes & no & yes & no & yes \\ 
	\hline 
	$lg(n!)$ & $lg(n^n)$ & yes & yes & no & no & no \\ 
	
\end{tabular} 

\section{Question 3: Problem 3-3-a}

$2^{2^{n+1}}> 2^{2^n}> (n+1)!> n! > e^n > n2^n \\> 2^n > \frac{3}{2}^n > n^{lglgn} = lgn^{lgn}>(lgn)!> n^3 \\> n^2 = 4^{lgn} > nlgn > 2^{lgn} = n > (2^{\frac{1}{2}})^{lgn} \\>2^{(2lgn)^{1/2}} >lg^2n > lg(n!)>lnn >(lgn)^{\frac{1}{2}} >ln(lnn)\\>2^{lg^*n}> lg^* n >lg^*lgn> lglg^*n> n^{\frac{1}{lgn}} >1$
\\
\\
Some procedure:\\
$n^n = 2^{nlgn} < 2^{2^{n}}\\
((2^{1/2})^{lgn}) = n^{1/2}\\
lg^2n = 2^{2lglgn}<2^{(2lgn)^{1/2}}<(2^{1/2})^{lgn}\\
n^{\frac{1}{lgn}} = 2^{\frac{lgn}{lgn}} = 2\\
4^{lgn} = n^{lg4} = n^2\\
n^{lglgn} = lgn^{lgn} =  e^{lnnlglgn} = 2^{lgnlglgn}>2^{(2lgn)^{(1/2)}}\\
n! > \frac{n^n}{e^n} = e^{nlnn - n} > e^{lnnlglgn}\\
lgn! = lgn^{1/2}\frac{lgn^{lgn}}{e^{lgn}}(1+\frac{1}{n}) < (lgn)^{lgn}\\
lnlnn = 2^{lglnlnn} > 2^{lg*n}$

\section{Question 4: Problem 3-4-c-d-e-f}

\subsection{c}
True.\\
$f(n) = O(g(n)) \rightarrow \exists c, n_{0}, \forall n > n_{0}, 1 \le f(n) \le cg(n) \\
\rightarrow 0 \le lg(f(n)) \le lg(cg(n)) = lgc + lg(g(n)) \\
\rightarrow \exists c^{'}, lgc + lg(g(n)) \le c^{'} lg(g(n))\\
\rightarrow lg(f(n)) = O(lg(g(n)))$
\subsection{d}
True.\\
$
f(n) = O(g(n)) \rightarrow \exists c, n_{0}, \forall n > n_{0}, 0 \le f(n) \le cg(n) \\
\rightarrow 1 \le 2^{f(n)} \le 2^{cg(n)} = 2^{c}2^{g(n)} \\
\rightarrow \exists c^{'} > 2^{c}, 2^{f(n)} \le c^{'}2^{g(n)}\\
\rightarrow 2^{f(n)} = O(2^{g(n)})$
\subsection{e}
False, consider any $f(x), \lim_{n \rightarrow \infty} f(x) < 1$, such as $e^{-x}$

\subsection{f}
True.\\
$
f(n) = O(g(n)) \rightarrow \exists c, n_{0}, \forall n > n_{0}, 0 \le f(n) \le cg(n)\\
\rightarrow \exists c^{'} = \frac{1}{c}, 0 \le c^{'}f(n) \le g(n)\\
\rightarrow g(n) = \Omega (f(n))
$
\section{Question 5: verify}
Proof: $ T(n) = O(n)$ \\
Suppose $\forall k<n, \exists c_{2}, T(k) \le c_{2}k - 10\\
\rightarrow T(n) = c_{2}\aleph n + c_{2}(1-\alpha) n - 20 + 10 \le c_{2}n - 10\\
\rightarrow T(n) = O(c_{2}n-10)\\
\rightarrow T(n) = O(n)\\$
\\
Proof: $ T(n) = \Omega(n)$ \\
Suppose $\forall k<n, \exists c_{1}, T(k) \ge c_{1}k\\
\rightarrow T(n) = c_{1}\aleph n + c_{1}(1-\alpha) n + 10 \ge c_{1}n\\
\rightarrow T(n) = \Omega(n)\\$
\\
$T(n) = O(n), T(n) = \Omega(n) \rightarrow T(n) = \Theta(n)$

\section{Question 6: solve and verify}
Notice that $TreeHeight=h = log_{\frac{3}{2}}n$\\
For branch $\Theta(n) = \sum _{1}^{h + 1} n(\frac{4}{3})^h = n\frac{(\frac{4}{3})^h - 1}{\frac{4}{3} - 1} = \Theta(n^{\frac{ln2}{ln3-ln2}})$\\
For leaf $\Theta(n) = 2^h = \Theta(n^{\frac{ln2}{ln3-ln2}})$\\
$\rightarrow T(n) = \Theta(n^{\frac{ln2}{ln3-ln2}}) = \Theta(n^{log_{\frac{3}{2}}2}) = \Theta(2^{log_{\frac{3}{2}}n})$\\
\\
Proof: $ T(n) = O(2^{log_{\frac{3}{2}}n} - 3n)$ \\
Suppose $\forall k<n, \exists c_{2}, T(k) \le c_{2}2^{log_{\frac{3}{2}}n} - 3n\\
\rightarrow T(n) = c_{2}2*2^{log_{\frac{3}{2}}\frac{2}{3}n} -4n + n \le c_{2}2^{log_{\frac{3}{2}}n} -3n\\
\rightarrow T(n) =  O(2^{log_{\frac{3}{2}}n} -3n)\\
\rightarrow T(n) =  O(2^{log_{\frac{3}{2}}n})\\$
\\
Proof: $ T(n) = \Omega(n^{log_{\frac{3}{2}}2})$ \\
Suppose $\forall k<n, \exists c_{1}, T(k) \ge c_{1}n^{log_{\frac{3}{2}}2}\\
\rightarrow T(n) = c_{1}2(\frac{2}{3}n)^{log_{\frac{3}{2}}2} + \frac{4}{3}n
 = c_{1}2*(\frac{3}{2})^{log_{\frac{3}{2}}2}*n^{log_{\frac{3}{2}}2} + \frac{4}{3}n = c_{1}n^{log_{\frac{3}{2}}2} + \frac{4}{3}n \ge c_{1}n^{log_{\frac{3}{2}}2} + n\\
\rightarrow T(n) = \Omega(n^{log_{\frac{3}{2}}2})\\$
\\
$T(n) =  O(2^{log_{\frac{3}{2}}n}), T(n) = \Omega(n^{log_{\frac{3}{2}}2}) \rightarrow T(n) = \Theta(n^{log_{\frac{3}{2}}2})$
\section{Question 7: solve and verify}

Notice that for iterative tree: \\
$\Theta (n^2) = 2T(\frac{1}{4}n) + n^2 \le T(n) \le 2T(\frac{1}{2}n) + n^2 = \Theta (n^2)\\$
\\
Proof: $T(n) = O(n^2)$, Suppose $\forall k < n, T(k) = O(k^2)\\
\rightarrow \exists c_{2}>\frac{16}{11}, T(k) \le c_{2}n^2, T(n) \le (\frac{5}{16}c_{2} + 1)n^{2} \le c_{2}n^2, c_{2}>\frac{16}{11} \\$
\\
Proof: $T(n) = \Omega(n^2)$, Suppose $\forall k < n, T(k) = \Omega(k^2)\\
\rightarrow \exists c_{1}<\frac{16}{11}, T(k) \ge c_{1}n^2, T(n) \ge (\frac{5}{16}c_{1} + 1)n^{2} \ge c_{1}n^2, c_{1}<\frac{16}{11} \\
\rightarrow T(n) = \Theta(n^2)$\\

\section{Question 8: solve}
Let $n = 2^m$, Then $T(2^m) = 9T(2^{\frac{m}{6}}) + m^2\\
\rightarrow S(m) = 9S(\frac{1}{6}m) + m^2\\$
From Branch: $S(m) = m^{log_{6}\frac{3}{2} + 2}$\\
From Leave: $S(m) = m^{log_{6}9}$\\
So, $S(m) = \Theta (m^{log_{6}\frac{3}{2} + 2}) \\
\rightarrow T(n) = T(2^m) = \Theta((lg(n))^{log_{6}\frac{3}{2} + 2})$

\section{Question 9: solve and justify}
\subsection{a}
For leaf $\Theta(n) = n^{log_{3}2}$\\
For branch, Notice that $n^{\frac{1}{2}lgn} < n^{\frac{1}{2} + \epsilon}\\
\rightarrow T(n) < S(n) 2S(\frac{1}{3}n) + n^{\frac{1}{2} + \epsilon}$\\
Notice that the branch complexity of $S(n) = n^{\frac{1}{2}+\epsilon} < n^{log_{3}2}$\\
$\rightarrow T(n)$ is dominated by leaf, $T(n) = \Theta(n^{log_{3}2})$


\subsection{b}
For branch $T(n) = \Theta(hn^2) = \Theta(lognn^2)$\\
For leaf $T(n) = \Theta(n^2)$\\
$\rightarrow T(n)$ is dominated by branch, $T(n) =\Theta(lognn^2)$

\subsection{c}
For leaf $T(n) = \Theta(4^{log_{2}n}) = \Theta(n^2)$\\
For branch, notice that $4*(\frac{1}{2})^{\frac{5}{2}} = 2^{-\frac{1}{2}} < 1, T(n) = \Theta(n^{\frac{5}{2}})$\\
$\rightarrow T(n)$ is dominated by branch, $T(n) =\Theta(n^{\frac{5}{2}})$

\subsection{d}
For branch $TreeHeight = h = \frac{n}{2}, T(n) = \frac{1}{2}\sum_{1}^{h + 1} \frac{1}{n} = \frac{1}{2}(lnn-ln2) = \Theta(lnn)$\\
For leaf $T(n) = \Theta(c)$\\
$\rightarrow T(n)$ is dominated by branch, $T(n) =\Theta(lnn)$


\end{document}
